\receta{Pan Payes de Levadura}

Rinde para 2 panes de 400g.

\begin{ingredientes}
\item 500g Harina de trigo panificable
\item 10g sal
\item 10g levadura fresca (ó 4g levadura seca)
\item 285g Agua
\end{ingredientes}
\preparacion
Poner todos los ingredientes en una batidora potente que tenga una herramienta para amasar pan. La batidora debe permanecer encendida hasta que se forme el gluten, momento en el cual la consistencia de la masa debe ser tal que no deje ingredientes adheridos al tasón y sea lisa y elastica. Si esta consistencia no se ha logrado solo es necesario continuar amasando.\\

Una vez obtenida la consistencia de chicle, se procede a realizar una fermentación de bloque durante 60 min, lo que significa dejar la masa en forma de bola en el tasón tapada.\\

Pasada la fermentación se parte la mezcla en 2 bolas de 500g. Se enharina una bandeja donde se vá fermentar el pan y se ponen alli las 2 bolas tapadas con un trapo en un lugar cálido por lo menoa 2 horas hasta que crescan un 75% de su volumen original. En caso de que las bolas se aplasten durante este proceso significa que el amasado no fue suficiente y es aconsejable repetir el proceso de amasado. Durante la espera precalentar el horno a $220^{\circ}C$ y se pone una bandeja metalica o cerámica donde se vá quemar el pan para que esta también se caliente.\\


Una vez pasado el tiempo de fermentación se le dá la vuelta a las bolas y con una cuchilla se les realiza una marca superficial en cruz sobre todo el hemisferio que mira hacia arriba; lo anterior permite que al momento de quemar la masa explosione por esta apertura.\\

Se transporta de la bandeja de fermentación a al horno cada uno de los  panes y despues usando un pulverizador de liquidos, se rocia un poco agua al interior del horno asegurandose que caiga sobre la bandrja caliente para generar vapor, este ayudará a que generación de la corteza del pán no sea tan rápida y la masa pueda seguir creciendo durante el quemado. El pán despues de aproximadamente 50 min deberia quedar dorado, sin embargo estos tiempos pueden variar dependiendo de la calidad el horno.\\

\paragraph{Alternativa de caldero holadés ( Dutch Oven )}: A cambio de usar una bandeja en el horno se puede usar en remplazo un caldero holadés de fundición con tapa incluida. La idea es meter el caldero en el horno durante el precalentamiento, al momento de quemar el pan sacar el caldero, poner la bola en su interior, rocear el agua al interior del caldero, volver a tapar y rápidamente de regreso el horno para que no se pierda el vapor.