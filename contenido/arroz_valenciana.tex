\receta{Arroz a la Valenciana}
\index{arrozvalenciana}

Rinde para 5 personas.


\begin{ingredientes}
\item Arroz blanco
\item Aceite de Oliva
\item Agua
\item Pimenton Rojo
\item Cebolla blanca
\item Tomate maduro
\item Queso Parmesano
\item Mantequilla
\item Sal
\item Pimienta Negra
\end{ingredientes}
\preparacion


En una olla se pone a sudar arroz blanco usando el agua y el arroz con el aceite de oliva sazonando con un poco de sal. El arrzón estará listo cuando el grano halla abierto y el agua se halla secado.\\
Por aparte en una sartén grande se pica el pimenton, cebolla y tomate en julianas y se ponen junto a la mantequilla a sofreir hasta que que reduzcan y hagan un hogo de color rojizo, este se adoba con sal y pimienta al gusto. En este punto se incorpora a la sartén el arróz que se fabricó anteriormente hasta obtener una mezcla homogenea y se agrega en caliente el queso parmesano.\\

Toda la mezcla se transpasa a un molde refractario plano en donde se recubre el arroz con una capa de queso parmezano para gratinar el plato y se pasa al horno a $232^{\circ}C$($450^{\circ}F$) por 30 min o hasta qeu la superficie de gratnado este dorada.
