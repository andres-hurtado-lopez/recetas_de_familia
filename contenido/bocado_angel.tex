\receta{Bocado de Angel}
A esta receta se aconseja acompañar con una “crema inglesa” que tambien está incluida en este recetario.

Rinde para 20 Porciones.\\

\begin{ingredientes}
\item 800gr leche condensada
\item 800gr crema de leche
\item 8 Claras de huevo
\item 480 gr Durasnos
\item $\rfrac{3}{4}$ taza de agua al clima
\item 21 gr gelatina sin sabor (colapiz)
\item 250 gr Ciruelas
\item Escencia de almendras
\end{ingredientes}
\preparacion

Se parten en julianas los duraznos y se pican las ciruelas en trozos gruesos reservandolos para aplicar al final.\\

Por aparte, en un recipiente se separan las claras de las llemas, llevan las claras a punto de nieve. Las yemas se pueden reservar para fabricar una crema inglesa que acompañe el postre. Todavia con la batidora en movimiento con las claras batiendose, se aplica la gelatina sin sabor previamente hidratada en agua tibia. Se agrega la leche condensada y la crema de leche formando mezcla homogenea.

En una fuente limpia se aplica una capa muy delgada de escencia de almendras vertiendo una pequeña cantidad sobre la fuente y luego esparciendola con una toalla de papel absorbente. Sobre la misma fuente se vierte una capa delgada de la mezcla preparada anteriormente y se aplican las frutas picadas en forma homogenea.\\

Se vierte otra capa mas gruesa de mezcla y se deja por 10 minutos a que la mezcla se aglutine en la fuente. Finalmente se aplica una última capa de frutas y se lleva la fuente a la heladera cubriendo la parte superior de la fuente con plastico por 6 horas para permitir la conformacion total de la gelatina en el postre.