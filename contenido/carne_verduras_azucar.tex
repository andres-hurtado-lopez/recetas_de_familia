\receta{Carne con Verduras y Azucar}
Esta receta salió de una de las revistas de mi abuelita, esta carne tiene algo especial que no se describir con palabras, a pesar de que el titulo suena algo horroroso por combinar carne con azúcar, puedo asegurar que su sabor es muy diferente a su titulo ya que de azúcar solo tiene una cucharada.

Esta receta es para 8 porciones de 230 g. Es excelente para se combinada con arroz blanco al vapor.

Valor nutricional por cada 100g de Carne con verduras :
\begin{itemize}
Energia: 153 kcal
Carbohidratos: 7.3g
Grasas: 8.2g
Proteínas: 12.7g
\end{itemize}

Rinde para X personas.

\begin{ingredientes}
\item 13g Azucar
\item 320g Pimenton Rojo y/o verde
\item 250g Cebolla partida solo en 4
\item 500g Champiñones laminados
\item 68g Fecula de maiz (Maizena)
\item 64g Salsa Soya (soja o como le digan)
\item 120g Aceite vegetal (yo usé canola)
\item 1000g Carne de Res mas o menos magra (Yo usé lomo)
\item 2 Tazas de agua
\item 2 Taza de arroz blanco
\end{ingredientes}
\preparacion
Por aparte en en una olla se pone a hervir el arroz como se hace usualmente hervido. Se pican en tiras los pimentones, se parten las cebollas en 4 separando las laminas de las cebollas. Estos ponen a saltear en un wok con 61g de aceite hasta que ablanden.\\

Se incorporan los champiñones laminados y de deja cocinar con el resto de las verduras. Una vez queden todas las verduras salteadas se sacan aparte del wok en una olla mas grande. Con la salsa soya, la fecula de maiz y el agua se hace una mezcla en un recipiente aparte.\\

En el wok se pone el aceite restante con la carne y se pone la carne a sofreir agregando lentamente la mezcla de salsa soya hecha anteriormente hasta lograr que la carne quede en su punto y la salsa espese, revisando constantemente el gusto de sal de la carne para adobarle en caso que haga falta.\\

Una vez lista la carne se incorporta con el resto de las verduras en la olla mas grande, es este punto se debe generar en a olla una salsa espesa de color café. A esta mezcla se le agrega el azucar y se deja reducir.\\

Se sirven porciones de 230g de carne con verdura y 200g de arroz.