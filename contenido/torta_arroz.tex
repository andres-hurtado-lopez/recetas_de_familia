\receta{Torta de Arroz}

Rinde para X personas.


\begin{ingredientes}
\item 2 tazas de arroz
\item Azafrán
\item Carne de cerdo
\item Sal
\item Tomates Maduros
\item Manteca de cerdo
\item Papas
\item Agua
\item Arvejas
\item parmesano
\item Huevos
\end{ingredientes}
\preparacion

Se suda 2 tazas de arroz común y corriente con un poco de azafrán. Para el guiso se corta carne de cerdo en cuadritos pequeños, se cocina con sal, mientras tanto se sofríen tomates maduros y azafrán con manteca de cerdo. (se sofríen los gordos que se le quitan a la carne) cuando ya se hallan sazonado los tomates se le pone la carne.

Aparte se pelan las papas y se parten en cuadritos pequeños y se cocinan en poca agua, cuando este se le echan a la carne hasta con el agua y se deja sazonar un rato procurando que no se seque pero tampoco muy aguado y por último la arvejas de tarro.

6 Claras ala nueve cuando estén se le echan las yemas con una pizca de sal. Estos huevos se le incorporan  Se engrasa el molde, se pone 1 capa de arroz así sucesivamente hasta terminar en arroz se echa el batido y se mete al horno. 


