\receta{Costillas Pegajosas de Kansas City}

Rinde para X personas.


\begin{ingredientes}
\item 2 Medias canales de costilla de cerdo que con el exeso de membrana y grasa removidos
\item Para adobar las costillas:
\begin{itemize}
\item 3 Cucharadas de paprika
\item 2 Cucharadas de azucar morena
\item 1 Cucharada de pimienta negra molida
\item 1 Cucharada de sal
\item $\rfrac{1}{4}$ de cucharadita de pimienta cayena
\end{itemize}
\item Para la salsa complementaria:
\begin{itemize}
\item 2 Cucharaditas de Aceite de cocina
\item 1 Cebolla cabezona picada
\item 4 tasas de caldo de pollo
\item 1 taza de cerveza de raiz (ponymalta)
\item 1 taza de vinagre de cidra o vinagre de manzana.
\item 1 taza de jarabe de maiz obscuro
\item $\rfrac{1}{2}$ taza de melaza de caña
\item $\rfrac{1}{2}$ taza de pasta de tomate
\item $\rfrac{1}{2}$ taza de ketchup
\item 2 cucharadas de mostaza cafe o mostaza antigua
\item 1 cucharada de salsa de tabasco
\item $\rfrac{1}{2}$ cucharadita de polvo de ajo
\item $\rfrac{1}{2}$ cucharadita de escencia de olor a ahumado. (opcional)
\end{itemize}
\end{ingredientes}
\preparacion

\emph{Para la Salsa}\\


\emph{Para las Costillas}\\

Tomar todos los ingredientes para adobar la costilla y mezclarlos en un recipiente aparte hasta que la mezcla quede uniforme.\\

Secar las costillas de la sangre que aun tenga impregnadas y a continuacion “sobar” las costillas con la preparacion de adobo mezclada de paso anterior hasta que queden totalmente impregnada por toda su superfecie asegurandose que el total de la mezcla quede sobre la superficie de las costillas, en caso de que no se vean como en las siguientes fotos aumentar las cantidades de la mezcla para adobo en iguales proporciones.\\

Precalentar el horno hasta alcanzar los $150^{\circ}C$ ($300^{\circ}F$) e introducir las costillas adobadas inicialmente por 45 min. Despues de debe dejar mas tiempo vijilando el color de la superficie de las costillas hasta asegurase que obtengan un color caramelo pero que no se queme la superficie. Una buena tecnica para poder hacer esto es disminur la temperatura del horno a $100^{\circ}C$($212^{\circ}F$) hasta lograr el color, lo cual lo puede hacer un poco demorado. Con esto las costillas quedan listas para ser servidas acompañadas de la salsa.\\
