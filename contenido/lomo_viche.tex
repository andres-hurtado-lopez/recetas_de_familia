\receta{Lomo Viche En Salsa de Champiñones}
Esta es una receta que preparo mi tia Martha Muñoz para el fin de año, que cosa tan rica, sirve para cualquier ocación especial:
\section*{Ingredientes}
\begin{itemize}
\setlength{\itemsep}{0pt}
\setlength{\parsep}{0pt} \setlength{\parskip}{0pt}
\item 4 lb de Lomo biche (solomo) en tajadas de 250gr
\item Mostaza
\item Salsa Inglesa
\item Sal
\item Aceite
\item Mantequilla
\item 1 Sobre mezcla para crema de champiñones
\item 1 Plato ondo Agua
\item 1 cubo magui
\item 1 copa de vino blanco
\item Sal
\item Champiñones tajados crudos
\item Pimienta de moler
\item 1 copa de vino blanco
\end{itemize}
\section*{Preparación}
Se prepara una mezcla la mostaza y la salsa inglesa adobada con sal al gusto.\\

Con la salsa preparada se impregnan las tajas de carne de lomo crudas de forma abundante que queden cubiertas sus dos caras.\\

En la estufa, en un perol ondo grande se le vierte aceite y mantequilla dejando que se derrita la mantequilla y acance una temperatura para freir la carne.\\

Las tajadas impregnadas de salsa se pasan por el perol de manera que frian cada una de sus caras hasta que queden selladas y no destilen sangre. El proceso anterior se puede hacer con la cantidad de tajadas maxima que pueda caber en el perol donde las carnes queden totalmente extendidas. Las carnes que han terminado el proceso se pueden poner aparte en un plato.\\

Para la base de la salsa se roma el aceite restante del proceso de sellado de la carne se deja calentar de forma que la sangre y salsa sean reducidos lo maximo posible intentando obtener un liquido mas o menos translucido.\\

En  un plato sopero con agua se agrega un sobre de mezcla de crema de champiñones, un cubo magui y se sazona con sal al gusto homogenizando la mezcla la cual se vierte en el aceite traslucido de proceso de sellado de la carne dejando hervir toda la mezcla para que reduzca y espese. Dependiendo de la consistencia deseada de la salsa se puede dejar mas tiempo al fuego agregando mas mezcla de crema de champiñones para aumentarla la viscozidad o agregar mas agua para disminuirla.\\

Al obtener la salsa deseada se agregan los champiños tajados en un pero abarte y se saltean un mantequilla, sazonandolos con pimiento negra molida. El resultado se incorpora a la salsa de champiñones controlando la viscocidad de la misma como se decribio en el paso anterior.\\

Finalmente se incorpora la carne sellada y una copa de vino que estaba por aparte dejando hervir mas tiempo toda la mezcla para que la carne absorba la salsa dejando hervir al gusto sin que se seque la carne. En caso de que la carne y la salsa se sequen mucho se puede agregar agua.

