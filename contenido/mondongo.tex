\receta{Sopa de Mondongo}

%Rinde para X personas.

\begin{ingredientes}
\item 1 Kg de mondongo
\item 1 Kg de carne de cerdo
\item 1 Kg de costilla de cerdo
\item 150g Arroz ($\rfrac{1}{2}$ taza)
\item 1 Yuca grande
\item 6 Papas Grandes
\item 500g de Alberjas
\item 3 Chorizos
\item Hogo
\item Ajo
\item Azafran
\item Cominos
\item Pimienta
\item Cebolla
\item Tomate
\item Limón
\item Color
\item Hogo (Ver receta pagina \pageref{hogo})
\end{ingredientes}
\preparacion
En una olla express se coloca el callo ò mondongo con agua y se deja cocinar hasta que este quede tierno. A continuación utilizando la misma agua resultante de le cocción se procede igualmente a cocinar la carne de cerdo hasta que esta también ablande. La carne y el mondongo se ponen aparte y en el caldo resultante se pone a cocinar el arroz, la papa, yuca y la alverja hasta que la mezcla espese.\\

En una sartèn se fríen los chorizos y se pican en trocitos al igual que el cerdo y el mondongo apartados anteriormente. Todo se pone junto en la olla inicial del caldo y se deja calar con hogo. Al finalizar se le agrega el comino.