\receta{Sopa de Mondongo}

Rinde para X personas.

\begin{ingredientes}
\item 1 Kg de mondongo
\item 1 Kg de carne de cerdo
\item 150g Arroz ($\rfrac{1}{2}$ taza)
\item $\rfrac{1}{2}$ Yuca grande
\item 3 Papas Grandes
\item 500g de Alberjas
\item 3 Chorizos
\item Hogo
\item Ajo
\item Azafran
\item Cominos
\item Pimienta
\item Cebolla larga
\item Tomate
\item Limón
\item Color
\item Cubo de caldo de gallina
\item Hogo (Ver receta pagina \pageref{hogo})
\end{ingredientes}
\preparacion
Con anticipacion, el mondongo se lava y se pone a remojar por 5 horas en jugo de limón, azafran , comino y hojas verdes de la cebolla larga que se vá a usar para el hogo. Mientras el mondongo está en remojo, se prepara una receta de hogo (ver página \pageref{hogo}), se pica la carne de cerdo en cuadrito y se ponene frerir los chorizos para luego picarlos en cuadritos.\\

Terminado el remojo del mondongo, se pica en cuadritos y se pone a cocinar en agua limpia hasta que dé el primer hevor en una olla express. Se decarta el agua hervida, reemplazandola por agua limpia y sazonadola con un cubo de caldo de gallina y sal para volver a llevar a hervor. Cuando el mondongo este blandito se saca de la olla y se reseva por aparte. Con el caldo restante, se pone a hervir en la misma olla express la carne de cerdo hasta que esté blandita e igualmente se reserva.\\

En una olla express se coloca el callo ò mondongo con agua y se deja cocinar hasta que este quede tierno. A continuación utilizando la misma agua resultante de le cocción se procede igualmente a cocinar la carne de cerdo hasta que esta también ablande. La carne y el mondongo se ponen aparte y en el caldo resultante se pone a cocinar el arroz, la papa, yuca y la alverja hasta que la mezcla espese.\\

En una sartèn se fríen los chorizos y se pican en trocitos al igual que el cerdo y el mondongo apartados anteriormente. Todo se pone junto en la olla inicial del caldo y se deja calar con hogo. Al finalizar se le agrega el comino.