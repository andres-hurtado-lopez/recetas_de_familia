\receta{Achuchas (Pepinos rellenos)}

Rinde para X personas.\\

\begin{ingredientes}
\item Para el relleno
\begin{itemize}
\item 1.5 Kg de Carne de cerdo
\item 5 Papas
\item 2 Huevos duros
\item Tomates maduros
\item Cebolla larga
\item Ajo
\item Cilantro
\item 2 Tazas de agua
\end{itemize}
\item Para la Salsa
\begin{itemize}
\item 6 Tomates
\item 1 Cebolla Cabezona
\item 2 Cebollas largas
\item 1 Ramo de perejil
\item 2 Rebanadas de pan
\item 4 Ajos
\item 1 Ramo de perejil
\end{itemize}
\end{ingredientes}
\preparacion

Para el relleno se licua en agua los tomates, ajo, cebolla larga, cilantro y en una sartén se hace con esto un hogo. Por aparte se pone a cocinar la carne hasta que ablande; para despues separa la carne del caldo y con éste se cocinan 
las papas. Por aparte se ponen a hervir los huevos hasta que queden duros y se pican en cuadritos finos con la carne y las papas. La carne, papas y huevo picado se incorporan al hogo a fuego lento en una sartén hasta que esté espese y se deja enfriar.\\

Para la salsa se pone a hervir en una olla con agua los tomates maduros partidos en cruz, la cebolla cabezona partida en cruz, cebolla larga sin picar, tajadas de pan, ramo de perejil y el ajo. Se deja enfriar para luego licuar y colar. Por ultimo se pone la mezcla en una olla con mantequilla, sal y pimienta reduciendo a fuego lento.\\

Se hierven los pepinos y se le sacan las pepas para despues rellenar y poner en un molde. Despues se bañan con la salsa y se llevan al horno.