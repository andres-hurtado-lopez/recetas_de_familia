\receta{Achuchas (Pepinos rellenos)}

Rinde para 10 personas.\\

\begin{ingredientes}
\item Para el relleno
\begin{itemize}
\item 1.5 Kg de Carne de cerdo
\item 5 Papas
\item 2 Huevos duros
\item Tomates maduros
\item Cebolla larga
\item Ajo
\item Cilantro
\end{itemize}
\item Para la Salsa
\begin{itemize}
\item 6 Tomates
\item 1 Cebolla Cabezona
\item 2 Cebollas largas
\item 1 Ramo de perejil
\item 2 Rebanadas de pan
\item 4 Ajos
\item 2 Tazas de agua
\end{itemize}
\item 10 Achuchas (pepinos de rellenar) Grandes
\end{ingredientes}
\preparacion

Se pone a cocinar la carne hasta que ablande; para después separar la carne del caldo y con éste se cocinan las papas. \\

Por aparte se ponen a hervir los huevos hasta que queden duros y se pican en cubitos finos con la carne y las papas. La carne, papas y huevo picado se incorporan al hogo a fuego lento en una sartén con esto queda el relleno.\\

Para la salsa se pone a hervir en una olla con agua los tomates maduros partidos en cruz, la cebolla cabezona partida en cruz, cebolla larga sin picar, tajadas de pan, ramo de perejil y el ajo. Se deja enfriar para luego licuar y colar. Por ultimo se pone la mezcla en una olla con mantequilla, sal y pimienta reduciendo a fuego lento.\\

Por aparte,se hierven las achuchas (pepinos) y se le sacan las pepas para después rellenar y poner en un molde. Finalmente se bañan con la salsa y se llevan al horno.