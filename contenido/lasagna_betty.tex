\receta{Lasagna Betty}
Esta receta la llama así porque desde que tengo uso de razón la conozco como la lasagna que hacia mi abuelita. Un gran problema de ésta receta es que en ningún lado había copia escrita, todo estaba en la cabeza de mis tías y mi abuela, razón por lo cual en una tarde de domingo, pedimos dirección técnica a mi mama (cosa que sabe hacer bastante bien) y logramos sacar la siguiente aproximación de la receta original. \\

Rinde para 30 personas.

\begin{ingredientes}
\item 3 kg Pechugas hervidas sin hueso
\item 3 kg Pierna de cerdo
\item 500 g Tocineta
\item 2 Cubitos de Caldo de gallina
\item 3 kg Tomates en trozos en latados
\item 800 g Pasta de tomate
\item 1 Cebolla de huevo
\item 4 Dientes de ajo
\item 1.5 kg Pasta de lasagna precocida
\item 1.5 kg Queso Holandes
\item 1.5 kg Queso parmesano
\item 345 g Crema de leche
\item Laurel Seco
\item Tomillo
\item Sal
\item Especias Italianas (Italian Seasoning)
\item 1 Manojo de Perejil picado.
\end{ingredientes}
\preparacion
Pone a cocinar el pollo y la carne de cerdo en una olla pitadora, se le hecha los cubitos de caldo de gallina, unas hojas de laurel y sal. Se pone a cocinar hasta que ablande. Cuando todo ablanda, se saca a un plato a enfriar, se deshebran las dos carnes. El caldo en que se cocinaron las carnes se reserva para mas adelante.\\

Se coloca en una sarten la tocineta a sofreir, se saca aparte en un plato dejando la grasa en la sartén. Se pica la cebolla y el ajo finamente y con la grasa de la tocineta se sofríen, si se necesita mas grasa se le adiciona aceite.\\

Se procesa los trozos de tomate para que queden en puré y se le agregan a la mezcla  de la cebolla en conjunto con la pasta de tomate, tomillo y laurel, dejándola en el fuego hasta que la mezcla tome un color rojo obscuro. Se le agregan el pollo y la carne de cerdo y se revuelve bien y poco a poco.\\

Se se le va agregando el caldo en que se cocino la carne hasta darle una consistencia de salsa, ya que de por si es muy seca y pastosa mezcla. Se deja a fuego lento, a medio tapar, para que todos los sabores se incorporen a la mezcla vigilando que no se seque, caso en el cual se le agrega el caldo.\\

Se le debe estar probando constantemente el sabor para verificar si requiere adiconarle sal a la mezcla. Para darle un toque final a la salsa se desmorona la tocineta, se le hecha un la crema de leche y el perejil picado.\\

En un molde refractario se pone una capa de salsa, encima una capa de pasta de lasagna remojada previamente en agua tibia y encima una capa de queso holandés y queso parmesano. Este proceso se repite varias veces armando capas hasta llegar al borde del molde. Lugar en el cual se debe acabar con una capa de queso,  es cubierta finamente con salsa y queso parmesano para gratinar la cubierta superior.\\

Todo se mete en horno precalentado a $350^\circ$F ($177^\circ$C) dejando hervir, que el queso parmesano quede derretido y el interior de la lasgna caliente.\\





