\receta{Lasagna (estilo Betty)}
Esta receta la llama así porque desde que tengo uso de razón la conozco como la lasagna que hacia mi abuelita. Un gran problema de ésta receta es que en ningún lado habia copia escrita, todo estaba en la cabeza de mis tias y mi abuela, razón por lo cual en una tarde de domingo, pedimos dirección técnica a mi mama (cosa que sabe hacer bastante bien) y logramos sacar la siguiente aproximación de la receta original. \\

Rinde para X personas.

\begin{ingredientes}
\item 2 pechugas
\item $\rfrac{1}{4}$ lb carne
\item $\rfrac{1}{4}$ lb tocineta
\item 2 Cubitos de Caldo de gallina
\item 800g de pasta de tomate fruko
\item 3 cebollas medianas
\item 4 dientes de ajo
\item 2 cajas de pasta de lasagna
\item Queso amarillo en rebanadas
\item Queso parmesano
\item Crema de leche
\item Laurel
\item Sal
\end{ingredientes}
\preparacion
Pone a cocinar el pollo y la carne de cerdo en una olla pitadora, se le hecha los cubitos de caldo de gallina, unas hojas de laurel y sal. Se pone a cocinar hasta que ablande. Cuando todo ablanda, se saca a un plato a enfriar, se deshebran las dos carnes.\\

Colocar en una sarten la tocineta y sofreir hasta que quede tostadita, se saca aparte en un plato dejando la grasa en la sartén. Con la grasa de la tocineta se sofríe la cebolla y el ajo. Si se necesita mas grasa se le adiciona aceite. Posteriormente se le agrega pasta de tomate, y tomillo, opcionalmente laurel, dejando a que el fuego la obscurezca. Cuando tome un color Rojo obscuro, se le agregan el pollo y la carne de cerdo, se revuelve bien y poco a poco se pe va agregando el caldo en que se cocino la carne hasta darle una consistencia de salsa, ya que de por si es muy seca y pastosa mezcla. Se deja a fuego lento, a medio tapar, para que todos los sabores se incorporen a la mezcla vigilando que no se seque, caso en el cual se le agrega el caldo. Se le debe estar probando constantemente si le hace falta agregar cualquiera de los ingredientes para que no se pierda un balance en el sabor o la textura. Para darle un toque final a la salsa se desmorona la tocineta y se le hecha un chorrito de crema de leche.\\

En un molde refractario se pone una capa de salsa, encima una capa de pasta de lasagna remojada previamente en agua tibia y encima una capa de queso amarillo. Este proceso se repite varias veces hasta llegar al borde del molde. Lugar en el cual se debe acabar con una capa de queso,  es cubierta finamente con salsa y queso parmesano para gratinar la cubierta superior. Todo se mete en horno precalentado a $350^\circ$F ($177^\circ$C) dejando hervir, que el queso parmesano quede derretido\\





