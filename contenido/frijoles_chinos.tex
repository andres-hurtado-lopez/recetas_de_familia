\receta{Frijoles Chinos}

Rinde para 6 personas.


\begin{ingredientes}
\item 700 g pocillos de fríjol verde 
\item 1 kg pechugas de pollo
\item 170 g Pimentón
\item 90 g Apio
\item 115 g Cebolla larga
\item 70 g Cebolla de huevo
\item 200 g Tomates maduros
\item 30 g Raíces chinas (Brotes de soya)
\item 2 Papas
\item 1 Cubo de caldo de gallina
\item 70 g Aceite vegetal de cocina
\item 2.5 L Agua
\item Sal
\item Pimienta Negra
\item Comino  
\item Tomillo
\item Salsa Soya
\end{ingredientes}
\preparacion


Se pica finamente por separado tomate, cebolla larga, cebolla de huevo, apio, pimentón y se reservan.

En la olla express se pone toda el agua a cocinar con las pechugas de pollo, 40g de cebolla larga picada, sal, pimienta y caldo de gallina. Cuando el pollo este cocinado, se saca el pollo y el agua que queda se reserva. Las pechugas se desmenuzan las pechugas se desmenuzan y se reservan también por aparte. 

Se hace un hogo utilizando: 75g cebolla larga, toda la cebolla de huevo, todo el apio, todo el pimentón, todo el tomate, sal, pimienta negra, mucho comino, tomillo y salsa soya.

Con el agua reservada de la cocción del pollo se ponen a cocinar los fríjoles verdes en la olla express. Cuando los fríjoles se hallan ablandado se le agrega el hogo preparado anteriormente y el pollo desmenuzado. Todo se deja calar por lo menos 10 minutos hasta obtener una consistencia ligeramente espesa y se le agregan la todas las raíces chinas.

Por aparte se ralla la papa y se fríen en tiras, la cuales al momento de servir los fríjoles se sirven encima como adorno.