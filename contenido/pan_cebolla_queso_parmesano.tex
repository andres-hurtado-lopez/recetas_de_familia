\receta{Pan de Cebolla Caramelizada y Queso Parmesano}
\begin{ingredientes}
\item 1 cebolla de huevo
\item 3 cucharadas de aceite de oliva
\item 650 g harina de trigo
\item 50 g Harina de alforfón (trigo sarraceno)
\item 252 g Harina de sémola
\item 50 g semilla de linaza molida
\item 2 cucharadas de especias italianas
\item 50 g de queso parmesano rayado
\item 700 g Agua
\item 30 g de yogur natural sin azúcar
\item 20 g sal
\item 275 g de masa madre hidratada al 80\% 
\end{ingredientes}

\preparacion

Pique finamente la cebolla y en un sartén poga a caramelizar en el aceite de oliva por 45 minutos. Deje después enfriar a temperatura ambiente y reserve.\\

Autolice (mezcla y dejar reposar) por 3 horas la harina de trigo, la harina de alforfón, la sémola, la linaza las especias italianas, el queso parmesano y el agua. Para hacer el proceso más fácil solo agregue 600g de agua a la mezcla y solo agregue los 100g restantes al final del proceso.\\

Mezcle el yogur, las cebollas caramelizadas, la sal y la masa madre. Espere a que su tamaño se triplique y añada a la mezcla principal de la receta.\\

En 3 oportunidades haga dobleces de la masa con separaciones de tiempo entre doblados de entre 20-30 minutos y deje la masa dobla en tamaño. Este proceso puede tardar entre 5 a 6 horas.\\

Divida la masa resultante de bolas de 729 g, deles forma, deje descansar por 15 minutos y vuelva a darles forma nuevamente para reajusta su forma de bola. Coloque las bolas sobre un recipiente plano y ponga en la nevera por 10 horas para que maduren.\\

La próxima mañana cocine en caldero holandés por 25 mintuos a $232^{\circ}C$ ($450^{\circ}F$), destape el caldero y después deje cocinar por 22 minutos a $218^{\circ}C$ ($425^{\circ}F$).
