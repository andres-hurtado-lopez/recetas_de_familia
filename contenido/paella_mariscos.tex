\receta{Paella de mariscos}
La receta es producto de la viajadera y tenaz estudio culinario de mi cansado tio Guillermo.\\

Rinde para 14 personas.


\begin{ingredientes}
\item Para El Caldo
\begin{itemize}
\item 1 Cabezas de pescado grande (corbina, mero, pargo rojo)
\item 1 rama de cebolla larga
\item Comino
\item Pimienta
\item Sal
\item Triguisar
\item 15 tazas de Agua
\end{itemize}
\item Para El Arroz
\begin{itemize}
\item 5 tazas de arroz parbolizado o arroz bomba espanol
\item Aceite de oliva
\item 800gr Cebolla cabezona
\item 600gr Tomate chonto
\item 3 pimentones en lajas
\item 1 sazonador de paella
\item 30 Gambas grandes (3 lb)
\item 1kg Anillos de calamar
\item 500 gr Mix de mariscos
\item 1kg pulpo
\item 1kg camarones
\item 1lb mejillones
\item 1lb almejas
\item Manteca de cerdo
\item Ajo
\item Vino blanco
\end{itemize}
\end{ingredientes}
\preparacion
Poner todos los ingredientes para el caldo en una olla a llama completa hasta que la carne suelte la sustancia, que no se pase de 40min.\\

Por aparte, se le retiran los intestinos a los langostinos, secandolos despues para que el agua no salpique cuando se pongan a sofreir. Es muy importante que todos los mariscos queden secos; Mi tio dice que para las Gambas no se debe hacer el procedimiento de lavado porque le “quita el sabor” según los españoles.\\

Se debe utilizar para la cocción una paellera con su respectiva estufa, ya que se necesita que tenga muy buena area para evaporar durante la coccion del arroz. Sobre esta paellera se sofrien las gambas en aceite de oliva hasta que estas dejen la grasa de color pardo y se sacan de la paellera dejando la grasa producto del sofrito como base para sofreir las verduras.\\

Con la grasa producto de las gambas se sofrie: el ajo, cebolla, pimenton, tomate, cuando todos esten sofritos se pasa a agregar el calamar y el pulpo, adiconando sal, pimienta y el sazonador. Finalmente se ponen las almejas fijandose que se abran.

Se procede a poner el arroz en la paellera para sofreirlo sobre todo el mix anteriormente preparado, la idea es que este tome los aromas de todos los mariscos. Como dato curioso los españoles tienen la superticion de que el arroz debe ser regado en forma de cruz, supongo que será para ``santificar'' el plato.\\

Cuando se vea un color pardo sobre el arroz se agrega el caldo de pescado, este servirá como un liquido sustituto del agua donde hervira el arroz para que el grano abra. En la mitad del proceso de coccion del arroz agregamos mejillones enterrandolos en el arroz en disposicion de anillos. Cuando se halla evaporado todo el caldo, se tapa con papel aluminio y se espera aproximadamente 20 minutos.