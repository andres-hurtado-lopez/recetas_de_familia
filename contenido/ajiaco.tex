\receta{Ajiaco}

Rinde para X personas.

\begin{ingredientes}
\item 3 Pechuga de pollo,con hueso y sin piel
\item 12 tazas de agua
\item 3 mazorcas de maíz fresco, 
\item cortadas en 2 piezas
\item $\rfrac{1}{4}$ de cucharadita de sal
\item Pimienta al gusto
\item 2 cubos de caldo de pollo
\item 3 cebollas largas
\item 2 dientes de ajo, picados
\item 3 cucharadas de cilantro fresco picado
\item 2 tazas de papa criolla
\item 3 papas blancas medianas, peladas y en rodajas
\item 3 papas rojas medianas, peladas y en rodajas
\item $\rfrac{1}{3}$ taza de guascas
\item 1 taza de crema espesa para servir
\item 1 taza de alcaparras para servir
\end{ingredientes}
\preparacion

En una olla grande, colocar el pollo, el maíz, el caldo de pollo, cilantro, la cebolla larga, ajo, sal y pimienta. Agregue el agua y poner a hervir, bajar el fuego a medio y cocinar durante unos 35 a 40 minutos, hasta que el pollo esté cocido y tierno. Retire el pollo y reservar.\\

Continuar cocinando maíz durante 30 minutos más. Retirar la cebolla larga y añadir las papas rojas, papas blancas, y las Guascas. Cocine durante 30 minutos más.\\

Destape la olla y agregue la papa criolla y cocine a fuego lento durante 15 a 20 minutos, sazonar con sal y pimienta.\\

Cortar la carne de pollo en trozos pequeños y poner de nuevo en la olla. Servir el Ajiaco caliente con alcaparras y crema de leche.